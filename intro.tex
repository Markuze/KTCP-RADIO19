\section{Introduction}
\T{CBN.} The CBN project ~\cite{Elastic} taps into an underutilized resource: the vast network infrastructure being built by the cloud providers~\cite{aws2018spending, gcp2018spending, microsoft2018spending}. A CBN customer, the \textit{client} connecting to any \textit{server} would traverse multiple \relays ~\cite{CDD}. In this paper, without loss of generality, we assume exactly two relays; in all figures the two assumed \relays are depicted as \rc, for the router connected to the client, and \rs, for the router connected to the server. In reality, these routers are just commodity VMs hosted by the cloud provider.

\T{TCP split.} TCP split is a well known technique for optimizing link utilization with TCP flows. For the past 20 years, TCP split has been used for wireless networks, satellite transmissions, LAN and WAN optimizations~\cite{chakravorty2003aggregation, le2015experiences,luglio2004,siracusano2016miniproxy,kernelsplit, Akamai_Radio}. 
Now, as part of the the CBN project, we are introducing TCP split into a new context of public clouds as a networking infrastructure. %\IK{repeating my comment, but again feel free to disregard: shouldn't we write here that beyond CBN, TCP split and the paper's results can be crucial to many NSBU projects, \eg to migrate VMs between datacenters? The potential to impact many VMware products is a significant criterion for VMware reviewers and readers.}